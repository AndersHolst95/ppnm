\documentclass[a4paper, 12pt]{article}
\usepackage[utf8]{inputenc}
\usepackage{graphicx}
\usepackage{amsmath}

\setlength{\parindent}{0pt}
\setlength{\parskip}{0pt}

\title{The errorfunction}
\author{Anders Holst Rasmussen}
\begin{document}
\maketitle
\section{Definition}
The error function (Gauss error function, erf) is defined as 
\begin{equation}
	\text{erf}(x) = \frac{2}{\sqrt{\pi}} \int^x_{0} e^{-t^2} dt,
\end{equation}

\begin{figure}
	\centering
	\input{plot.tex}
	\caption{Plot of the error function with tabulated values}
	\label{fig:erf}
\end{figure}
and can be seen on figure \ref{fig:erf}, along with some of the tabulated values. \\
This function is generally used in probabillity and statistics to describe diffusion. 

\section{Usage}
In statistics, the errorfunction is used as a probabillity for a single measurement. When some measurements can be described as a normal distribution with a standard deviation $\sigma$ and expected value 0, $\text{erf}\left(\frac{a}{\sigma\sqrt{2}} \right)$ is the probabillity that a single measurement lies between -a and +a. This is very useful in determining bit error rate of a digital communication system. 

\section{Properties}
The errorfunction is an odd function, meaning that $\text{erf}(-z) = -\text{erf}(z)$.\\
The derivative of the errorfunction is
\begin{equation}
	\frac{d}{dz}\text{erf}(z) = \frac{2}{\sqrt{\pi}} e^{-z^2}.
\end{equation}


\end{document}
